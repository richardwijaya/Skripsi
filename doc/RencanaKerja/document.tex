\documentclass[a4paper,twoside]{article}
\usepackage[T1]{fontenc}
\usepackage[bahasa]{babel}
\usepackage{graphicx}
\usepackage{graphics}
\usepackage{float}
\usepackage[cm]{fullpage}
\pagestyle{myheadings}
\usepackage{etoolbox}
\usepackage{setspace} 
\usepackage{lipsum} 
\setlength{\headsep}{30pt}
\usepackage[inner=2cm,outer=2.5cm,top=2.5cm,bottom=2cm]{geometry} %margin
% \pagestyle{empty}

\makeatletter
\renewcommand{\@maketitle} {\begin{center} {\LARGE \textbf{ \textsc{\@title}} \par} \bigskip {\large \textbf{\textsc{\@author}} }\end{center} }
\renewcommand{\thispagestyle}[1]{}
\markright{\textbf{\textsc{AIF184001 \textemdash Rencana Kerja Skripsi \textemdash Sem. Genap 2019/2020}}}

\newcommand{\HRule}{\rule{\linewidth}{0.4mm}}
\renewcommand{\baselinestretch}{1}
\setlength{\parindent}{0 pt}
\setlength{\parskip}{6 pt}

\onehalfspacing
 
\begin{document}

\title{\@judultopik}
\author{\nama \textendash \@npm} 

%tulis nama dan NPM anda di sini:
\newcommand{\nama}{Richard Wijaya}
\newcommand{\@npm}{2016730014}
\newcommand{\@judultopik}{Virtual Jogging App untuk Google Cardboard} % Judul/topik anda
\newcommand{\jumpemb}{1} % Jumlah pembimbing, 1 atau 2
\newcommand{\tanggal}{06/02/2020}

% Dokumen hasil template ini harus dicetak bolak-balik !!!!

\maketitle

\pagenumbering{arabic}

\section{Deskripsi}
Pada zaman modern ini, mayoritas orang menekuni berbagai bidang profesi yang tak terhitung banyaknya. Hal ini menyebabkan kesulitan pengaturan waktu untuk berolahraga. Salah satu aktivitas olahraga yang paling mudah dan tidak memerlukan gerakan yang sulit adalah berlari, namun kegiatan ini memerlukan lahan yang cukup besar dan lebih menyenangkan jika dilakukan di luar rumah. Solusi untuk menghemat lahan agar berlari dapat dilakukan di rumah adalah {\it treadmill}, namun masalah lingkungan yang monoton dan membosankan di dalam rumah membuat orang enggan untuk melakukan aktivitas berlari. 

Pada skripsi ini, akan dibuat sebuah perangkat lunak yang dapat menampilkan simulasi aktivitas berlari pada lingkungan yang diinginkan saat berlari di {\it treadmill}. Dengan menggunakan perangkat lunak tersebut, orang yang berlari dapat menikmati pemandangan yang dipilih saat berlari di dalam rumah sehingga merasa seperti berlari di lingkungan yang dipilih tersebut.

Teknologi yang dapat dimanfaatkan untuk membuat aplikasi VR untuk berlari adalah Google Cardboard dan sensor perangkat bergerak, dan untuk {\it Application Programming Interface} (API) yang digunakan adalah {\it Google Streetview} API dan {\it Google Directions} API.

\section{Rumusan Masalah}
Rumusan masalah yang ada pada skripsi ini adalah:
\begin{itemize}
	\item Bagaimana memanfaatkan Google VR SDK for Android untuk menampilkan gambar dengan perangkat VR?
	\item Bagaimana menampilkan hasil dari Google StreetView API dalam bentuk VR?
	\item Bagaimana mengintegrasikan Google Directions API, gambar Google StreetView dan Google VR dalam perangkat lunak virtual jogging?
\end{itemize}


\section{Tujuan}
Pada skripsi ini, hal-hal yang coba untuk dicapai adalah :
\begin{itemize}
	\item Menggunakan Google VR SDK for Android untuk menampilkan gambar dengan {\it Google Cardboard}.
	\item Menampilkan hasil gambar dari Google StreetView API pada {\it Google Cardboard}.
	\item Mengintegrasikan Google Directions API, gambar dari Google StreetView dan Google VR (Cardboard) dalam perangkat lunak {\it virtual jogging}.
\end{itemize}

%Tuliskan tujuan dari topik skripsi yang anda ajukan. Tujuan penelitian biasanya berkaitan erat dengan pertanyaan yang diajukan di bagian rumusan masalah. Gunakan itemize seperti contoh di bagian Deskripsi Perangkat Lunak.


\section{Deskripsi Perangkat Lunak}
Perangkat lunak akhir yang akan dibuat memiliki fitur minimal sebagai berikut :
\begin{itemize}
	\item Pengguna dapat memilih pemandangan yang diinginkan dari yang telah ditawarkan.
	\item Pengguna dapat melihat pemandangan yang dipilih dalam {\it Google Cardboard}.
	\item Pengguna dapat merasakan bahwa pemandangan bergerak seolah-olah pengguna sedang berlari saat menggunakan perangkat lunak.
	\item Pengguna dapat merasakan seperti berjalan di jalan yang ada secara fisik di tempat yang dipilih.
\end{itemize}

\section{Detail Pengerjaan Skripsi}
Bagian-bagian pekerjaan skripsi ini adalah sebagai berikut :
\begin{enumerate}
	\item Mempelajari {\it Google StreetView, Cardboard} (Google VR SDK), dan {\it Directions}  API.
	\item Menampilkan {\it StreetView} dalam {\it Google Cardboard}.
	\item Mempelajari pergerakan berlari dan bagaimana mempengaruhi sensor perangkat bergerak. 
	\item Mengimplementasikan rute dari pelari di aplikasi dengan {\it Google Directions} API.
	\item Mengimplementasikan pergerakkan lingkungan dalam aplikasi dengan memanfaatkan sensor agar terlihat seperti sedang berlari.
	\item Membuat antarmuka untuk pengguna.
\end{enumerate}


\section{Rencana Kerja}
Rincian capaian yang direncanakan di Skripsi 1 adalah sebagai berikut:
\begin{enumerate}
\item Berhasil menampilkan gambar {\it StreetView} dalam {\it Google Cardboard}. 
\item Mengetahui bagian sensor yang menerima rangsang saat pergerakan berlari.
\end{enumerate}

Sedangkan yang akan diselesaikan di Skripsi 2 adalah sebagai berikut:
\begin{enumerate}
\item Antarmuka pengguna dapat membantu pengguna menggunakan aplikasi.
\item Aplikasi dapat membuat pengguna merasakan lingkungan yang ditampilkan {\it Google Cardboard} bergerak seperti saat berlari saat pengguna berlari.  
\item Dokumen Skripsi sudah lengkap.
\end{enumerate}


\vspace{7cm}
\centering Bandung, \tanggal\\
\vspace{2cm} \nama \\ 
\vspace{1cm}

Menyetujui, \\
\ifdefstring{\jumpemb}{2}{
\vspace{1.5cm}
\begin{centering} Menyetujui,\\ \end{centering} \vspace{0.75cm}
\begin{minipage}[b]{0.45\linewidth}
% \centering Bandung, \makebox[0.5cm]{\hrulefill}/\makebox[0.5cm]{\hrulefill}/2013 \\
\vspace{2cm} Nama: \makebox[3cm]{\hrulefill}\\ Pembimbing Utama
\end{minipage} \hspace{0.5cm}
\begin{minipage}[b]{0.45\linewidth}
% \centering Bandung, \makebox[0.5cm]{\hrulefill}/\makebox[0.5cm]{\hrulefill}/2013\\
\vspace{2cm} Nama: \makebox[3cm]{\hrulefill}\\ Pembimbing Pendamping
\end{minipage}
\vspace{0.5cm}
}{
% \centering Bandung, \makebox[0.5cm]{\hrulefill}/\makebox[0.5cm]{\hrulefill}/2013\\
\vspace{2cm} Pascal Alfadian Nugroho
%\Nama: makebox[3cm]{\hrulefill}\\ Pembimbing Tunggal
}
\end{document}

