%%versi 2 (8-10-2016)
%\chapter{Hasil Eksperimen}
%\label{lamp:B}
\chapter{Kode Program Aplikasi \textit{Jogging}}
\label{lamp:B}

\begin{lstlisting}[language=Java,caption=MainActivity.java]
package id.ac.unpar.informatika.homerunner;


import android.app.Activity;
import android.os.Bundle;
import android.view.View;
import android.widget.Button;
import android.widget.EditText;
import android.widget.TextView;
import android.widget.Toast;

public class MainActivity extends Activity implements View.OnClickListener {

    protected EditText originET, destET;
    protected TextView oriTV, destTV;
    protected Button startButton;

    @Override
    protected void onCreate(Bundle savedInstanceState) {
        super.onCreate(savedInstanceState);
        setContentView(R.layout.activity_main);

        startButton = (Button) findViewById(R.id.startBtn);
        originET = (EditText) findViewById(R.id.originET);
        destET = (EditText) findViewById(R.id.destET);
        oriTV = (TextView) findViewById(R.id.originTV);
        destTV = (TextView) findViewById(R.id.destTV);

        oriTV.setText("Origin");
        destTV.setText("Destination");

        startButton.setText("Start Running");
        startButton.setOnClickListener(this);
    }

    @Override
    public void onClick(View view) {
        if(view.getId() == startButton.getId()){

            if (originET.getText().toString().equals("") || destET.getText().toString().equals("")) {
                Toast.makeText(getBaseContext(),"Origin or Destination Text Box can't be Empty",
                        Toast.LENGTH_SHORT).show();
            } else {

                DirectionsLoader dirLoader = new DirectionsLoader(this);

                String dirURL = "https://maps.googleapis.com/maps/api/directions/json?" +
                        "mode=walking&origin=" + originET.getText().toString() + "&destination=" +
                        destET.getText().toString() + "&key=" + getString(R.string.key);

                originET.setText("");
                destET.setText("");

                dirLoader.execute(dirURL);
            }
        }
    }
}

}
\end{lstlisting}

\begin{lstlisting}[language=Java,caption=DirectionsLoader.java]
package id.ac.unpar.informatika.homerunner;

import android.app.Activity;
import android.content.Intent;
import android.os.AsyncTask;
import java.io.BufferedReader;
import java.io.InputStream;
import java.io.InputStreamReader;
import java.net.HttpURLConnection;
import java.net.URL;

public class DirectionsLoader extends AsyncTask<String,Void,Void> {

    protected String jsonText;
    protected Activity activity;

    public DirectionsLoader(Activity activity){
        this.activity = activity;
        jsonText = "";
    }

    @Override
    protected Void doInBackground(String... strings) {
        try {
            URL url = new URL(strings[0]);
            HttpURLConnection conn = (HttpURLConnection) url.openConnection();
            conn.connect();
            InputStream inputStream = conn.getInputStream();

            InputStreamReader inputStreamReader = new InputStreamReader(inputStream);
            BufferedReader bufferedReader = new BufferedReader(inputStreamReader);

            String curLine = bufferedReader.readLine();;

            while(curLine != null){
                jsonText += curLine;
                curLine = bufferedReader.readLine();

            }

            conn.disconnect();

        } catch (Exception ex) {}

        return null;
    }

    @Override
    protected void onPostExecute(Void aVoid) {
        DirectionsExtractor dirExtractor = new DirectionsExtractor(jsonText);
        dirExtractor.extractJSONDir();

        Intent intent = new Intent(activity, VrActivity.class);

        StreetViewLoader streetViewLoader = new StreetViewLoader(activity, intent, dirExtractor.arrSteps);
        streetViewLoader.execute();
    }
}
\end{lstlisting}

\begin{lstlisting}[language=Java,caption=DirectionsExtractor.java]
package id.ac.unpar.informatika.homerunner;

import org.json.JSONArray;
import org.json.JSONObject;
import java.io.BufferedReader;
import java.io.StringReader;
import java.util.ArrayList;

public class DirectionsExtractor {

    protected String jsonText;
    protected ArrayList<JSONObject> arrSteps;

    public DirectionsExtractor(String jsonText){
        this.jsonText = jsonText;
        arrSteps = new ArrayList<>();
    }

    public void extractJSONDir(){
        String jsonTemp = "";
        JSONObject jsonDir;

        try {
            BufferedReader bufferedReader = new BufferedReader(new StringReader(jsonText));
            String curLine = bufferedReader.readLine();

            while (curLine != null){
                jsonTemp += curLine;
                curLine = bufferedReader.readLine();
            }

            jsonDir = new JSONObject(jsonTemp);

            JSONArray jsonArrRoute = jsonDir.getJSONArray("routes");

            for(int i = 0; i < jsonArrRoute.length(); i++){
                JSONObject jsonRoute = jsonArrRoute.getJSONObject(i);

                JSONArray jsonArrLegs = jsonRoute.getJSONArray("legs");

                for(int j = 0; j < jsonArrLegs.length(); j++){
                    JSONObject jsonLegs = jsonArrLegs.getJSONObject(j);


                    JSONArray jsonArrSteps = jsonLegs.getJSONArray("steps");

                    for(int k = 0; k < jsonArrSteps.length(); k++){
                        arrSteps.add(jsonArrSteps.getJSONObject(k));
                    }
                }
            }
        }catch (Exception ex){

        }
    }
}
\end{lstlisting}

\begin{lstlisting}[language=Java,caption=StreetViewLoader.java]
package id.ac.unpar.informatika.homerunner;

import android.app.Activity;
import android.content.Intent;
import android.graphics.Bitmap;
import android.graphics.BitmapFactory;
import android.graphics.Canvas;
import android.os.AsyncTask;
import org.json.JSONObject;
import java.io.File;
import java.io.FileOutputStream;
import java.io.InputStream;
import java.net.URL;
import java.util.ArrayList;

public class StreetViewLoader extends AsyncTask<Void, Void, Void> {

    protected Bitmap streetViewBitmap;
    protected Activity activity;
    protected Intent intent;
    protected ArrayList<JSONObject> arrSteps;
    protected int imgCount;

    public StreetViewLoader(Activity activity, Intent intent, ArrayList<JSONObject> arrSteps) {
        this.intent = intent;
        this.activity = activity;
        this.arrSteps = arrSteps;
        this.imgCount = 0;
    }

    @Override
    protected Void doInBackground(Void... aVoid) {
        for(int i = 0; i < arrSteps.size(); i++){
            int length = 4;

            boolean isEnd = (i == arrSteps.size() - 1 ? true : false);

            createStreetViewImage(generateStreetViewURL(length, true));
            publishProgress();

            if(isEnd) {
                createStreetViewImage(generateStreetViewURL(length,false));
                publishProgress();
            }
        }

        return null;
    }

    @Override
    protected void onProgressUpdate(Void... values) {
        imgCount++;
    }

    @Override
    protected void onPostExecute(Void aVoid) {
        intent.putExtra("length", imgCount);

        int[] stepsDistance = new int[arrSteps.size()];
        for (int i = 0; i < arrSteps.size(); i++){
            try{
                stepsDistance[i] = arrSteps.get(i).getJSONObject("distance").getInt("value");
            } catch (Exception ex){}
        }

        intent.putExtra("stepDistances", stepsDistance);

        activity.startActivity(intent);
    }

    protected String[] generateStreetViewURL(int svUrlLength, boolean isStart){
        int heading = 0;

        String[] urlArr = new String[svUrlLength];

        String location = (isStart ? getLocation("start_location") : getLocation("end_location"));

        String svTempURL = "https://maps.googleapis.com/maps/api/streetview?size=600x300&location="+
                location +"&key="+ activity.getString(R.string.key) + "&heading=";

        for(int i = 0 ; i < svUrlLength ; i++){
            urlArr[i] = svTempURL + heading;
            heading += 90;
        }

        return urlArr;
    }

    public String generateFilePath() {
        return ("streetview" + imgCount + ".png");
    }

    public String getLocation(String startOrEndKey){
        double startLoc = 0.0;
        double endLoc = 0.0;
        try {
            startLoc = arrSteps.get(imgCount).getJSONObject(startOrEndKey).getDouble("lat");
            endLoc = arrSteps.get(imgCount).getJSONObject(startOrEndKey).getDouble("lng");
        } catch (Exception ex){}

        return startLoc + "," + endLoc;
    }

    public void createStreetViewImage(String[] urls){
        Bitmap[] bitmaps = new Bitmap[urls.length];
        for (int j = 0; j < urls.length; j++) {
            try {
                URL url = new URL(urls[j]);
                InputStream input = url.openStream();
                bitmaps[j] = BitmapFactory.decodeStream(input);
            } catch (Exception ex) {}
        }

        int width = bitmaps[0].getWidth();
        int height = bitmaps[0].getHeight();

        width *= bitmaps.length;

        Bitmap allBitmap = Bitmap.createBitmap(width, height, Bitmap.Config.ARGB_8888);
        Canvas canvas = new Canvas(allBitmap);
        int left = 0;
        for (int j = 0; j < bitmaps.length; j++) {
            left = (j == 0 ? 0 : left + bitmaps[j].getWidth());
            canvas.drawBitmap(bitmaps[j], left, 0, null);
        }
        streetViewBitmap = allBitmap;

        try {
            File file = new File(activity.getCacheDir(), generateFilePath());
            if (file.exists()) {
                file.delete();
                file = new File(activity.getCacheDir(), generateFilePath());
            }
            FileOutputStream fOS = new FileOutputStream(file);
            streetViewBitmap.compress(Bitmap.CompressFormat.PNG, 0, fOS);
            fOS.flush();
            fOS.close();
        } catch (Exception e) {}
    }

}
\end{lstlisting}

\begin{lstlisting}[language=Java,caption=VrActivity.java]
package id.ac.unpar.informatika.homerunner;

import android.content.Context;
import android.hardware.Sensor;
import android.hardware.SensorEvent;
import android.hardware.SensorEventListener;
import android.hardware.SensorManager;
import android.opengl.GLES20;
import android.opengl.Matrix;
import android.os.Bundle;
import android.util.Log;

import com.google.vr.ndk.base.Properties;
import com.google.vr.ndk.base.Properties.PropertyType;
import com.google.vr.ndk.base.Value;
import com.google.vr.sdk.base.AndroidCompat;
import com.google.vr.sdk.base.Eye;
import com.google.vr.sdk.base.GvrActivity;
import com.google.vr.sdk.base.GvrView;
import com.google.vr.sdk.base.HeadTransform;
import com.google.vr.sdk.base.Viewport;

import java.io.File;
import java.io.IOException;
import java.util.ArrayList;

import javax.microedition.khronos.egl.EGLConfig;

public class VrActivity extends GvrActivity implements GvrView.StereoRenderer, SensorEventListener {

  private static final float Z_NEAR = 0.01f;
  private static final float Z_FAR = 10.0f;

  private static final float[] POS_MATRIX_MULTIPLY_VEC = {0.0f, 0.0f, 0.0f, 1.0f};
  private static final float[] FORWARD_VEC = {0.0f, 0.0f, -1.0f, 1.f};

  private static final float DEFAULT_FLOOR_HEIGHT = -1.6f;

  private static final String[] OBJECT_VERTEX_SHADER_CODE =
          new String[] {
                  "uniform mat4 u_MVP;",
                  "attribute vec4 a_Position;",
                  "attribute vec2 a_UV;",
                  "varying vec2 v_UV;",
                  "",
                  "void main() {",
                  "  v_UV = a_UV;",
                  "  gl_Position = u_MVP * a_Position;",
                  "}",
          };
  private static final String[] OBJECT_FRAGMENT_SHADER_CODE =
          new String[] {
                  "precision mediump float;",
                  "varying vec2 v_UV;",
                  "uniform sampler2D u_Texture;",
                  "",
                  "void main() {",
                  "  // The y coordinate of this sample's textures is reversed compared to",
                  "  // what OpenGL expects, so we invert the y coordinate.",
                  "  gl_FragColor = texture2D(u_Texture, vec2(v_UV.x, 1.0 - v_UV.y));",
                  "}",
          };

  protected int objectProgram;

  protected int objectPositionParam;
  protected int objectUvParam;
  protected int objectModelViewProjectionParam;

  private float[] camera;
  private float[] view;
  private float[] headView;
  private float[] modelViewProjection;
  private float[] modelView;

  private float[] modelRoom;

  private float[] headRotation;

  private Properties gvrProperties;

  protected TexturedMesh room;
  protected ArrayList<Texture> roomTex;
  protected Texture finishedTexture;

  private final Value floorHeight = new Value();

  private SensorManager sensorManager;
  private Sensor stepDetector;

  private static final int DISTANCE_PER_STEP = 100;

  private int distanceElapsed;
  private int curStepIndex;
  private int curStepDistance;
  private int[] stepsDistance;

  @Override
  public void onCreate(Bundle savedInstanceState) {
    super.onCreate(savedInstanceState);

    sensorManager = (SensorManager) this.getSystemService(Context.SENSOR_SERVICE);
    stepDetector = sensorManager.getDefaultSensor(Sensor.TYPE_STEP_DETECTOR);

    sensorManager.registerListener(this, stepDetector, SensorManager.SENSOR_DELAY_FASTEST);

    distanceElapsed = 0;
    curStepIndex = 0;

    roomTex = new ArrayList<>();

    stepsDistance = getIntent().getIntArrayExtra("stepDistances");

    curStepDistance = stepsDistance[curStepIndex];

    initializeGvrView();

    camera = new float[16];
    view = new float[16];
    modelViewProjection = new float[16];
    modelView = new float[16];
    
    headRotation = new float[4];
    modelRoom = new float[16];
    headView = new float[16];

    if (sensorManager.getDefaultSensor(Sensor.TYPE_STEP_DETECTOR) != null) {
      Log.d("Step Detector", "This device has it");
    } else {
      Log.d("Step Detector", "This device don't have it");
    }
  }

  public void initializeGvrView() {
    setContentView(R.layout.activity_vr);

    GvrView gvrView = findViewById(R.id.gvr_view);

    gvrView.setRenderer(this);
    gvrView.setTransitionViewEnabled(true);

    gvrView.enableCardboardTriggerEmulation();

    if (gvrView.setAsyncReprojectionEnabled(true)) {
      // Async reprojection decouples the app framerate from the display framerate,
      // allowing immersive interaction even at the throttled clockrates set by
      // sustained performance mode.
      AndroidCompat.setSustainedPerformanceMode(this, true);
    }

    setGvrView(gvrView);
    gvrProperties = gvrView.getGvrApi().getCurrentProperties();
  }

  @Override
  public void onPause() {
    super.onPause();
  }

  @Override
  public void onResume() {
    super.onResume();
  }

  @Override
  public void onRendererShutdown() {
    floorHeight.close();
  }

  @Override
  public void onSurfaceChanged(int width, int height) {}

  
  @Override
  public void onSurfaceCreated(EGLConfig config) {
    GLES20.glClearColor(0.0f, 0.0f, 0.0f, 0.0f);

    objectProgram = Util.compileProgram(OBJECT_VERTEX_SHADER_CODE, OBJECT_FRAGMENT_SHADER_CODE);

    objectPositionParam = GLES20.glGetAttribLocation(objectProgram, "a_Position");
    objectUvParam = GLES20.glGetAttribLocation(objectProgram, "a_UV");
    objectModelViewProjectionParam = GLES20.glGetUniformLocation(objectProgram, "u_MVP");

    Util.checkGlError("Object program params");

    Matrix.setIdentityM(modelRoom, 0);
    Matrix.translateM(modelRoom, 0, 0, DEFAULT_FLOOR_HEIGHT, 0);

    int imgLength = getIntent().getIntExtra("length",0);

    try {
      finishedTexture = new Texture(this, "finish_image.png", false);
    } catch (IOException e){

    }

    for(int i = 0; i < imgLength; i++) {
      try {
        room = new TexturedMesh(this, "Room.obj", objectPositionParam, objectUvParam);
        roomTex.add(new Texture(this, "streetview" + i + ".png", true));
      } catch (IOException e) {}
    }
  }
  
  @Override
  public void onNewFrame(HeadTransform headTransform) {
    // Build the camera matrix and apply it to the ModelView.
    Matrix.setLookAtM(camera, 0, 0.0f, 0.0f, 0.0f, 0.0f, 0.0f, -1.0f, 0.0f, 1.0f, 0.0f);

    if (gvrProperties.get(PropertyType.TRACKING_FLOOR_HEIGHT, floorHeight)) {
      // The floor height can change each frame when tracking system detects a new floor position.
      Matrix.setIdentityM(modelRoom, 0);
      Matrix.translateM(modelRoom, 0, 0, floorHeight.asFloat(), 0);
    } // else the device doesn't support floor height detection so DEFAULT_FLOOR_HEIGHT is used.

    headTransform.getHeadView(headView, 0);

    headTransform.getQuaternion(headRotation, 0);
    Util.checkGlError("onNewFrame");
  }

  @Override
  public void onDrawEye(Eye eye) {
    GLES20.glEnable(GLES20.GL_DEPTH_TEST);
    // The clear color doesn't matter here because it's completely obscured by
    // the room. However, the color buffer is still cleared because it may
    // improve performance.
    GLES20.glClear(GLES20.GL_COLOR_BUFFER_BIT | GLES20.GL_DEPTH_BUFFER_BIT);

    // Apply the eye transformation to the camera.
    Matrix.multiplyMM(view, 0, eye.getEyeView(), 0, camera, 0);

    // Build the ModelView and ModelViewProjection matrices
    // for calculating the position of the target object.
    float[] perspective = eye.getPerspective(Z_NEAR, Z_FAR);

    Matrix.multiplyMM(modelViewProjection, 0, perspective, 0, modelView, 0);

    // Set modelView for the room, so it's drawn in the correct location
    Matrix.multiplyMM(modelView, 0, view, 0, modelRoom, 0);
    Matrix.multiplyMM(modelViewProjection, 0, perspective, 0, modelView, 0);
    drawRoom();
  }

  @Override
  public void onFinishFrame(Viewport viewport) {
    File cacheDir = getCacheDir();
    emptyDir(cacheDir);
  }

  public static boolean emptyDir(File dir) {
    if (dir.isDirectory()) {
      String[] children = dir.list();
      for (int i = 0; i < children.length; i++) {
        if (!emptyDir(new File(dir, children[i])))
          return false;
      }
    }
    return dir.delete();
  }
  
  public void drawRoom() {
    GLES20.glUseProgram(objectProgram);
    GLES20.glUniformMatrix4fv(objectModelViewProjectionParam, 1, false, modelViewProjection, 0);
    if (curStepIndex >= roomTex.size())
      finishedTexture.bind();
    else
      roomTex.get(curStepIndex).bind();
    room.draw();
    Util.checkGlError("drawRoom");
  }

  @Override
  public void onCardboardTrigger() {}

  @Override
  public void onSensorChanged(SensorEvent sensorEvent) {
    if(sensorEvent.sensor == this.stepDetector) {
      if (curStepIndex <= stepsDistance.length) {
        distanceElapsed += DISTANCE_PER_STEP;
        if (distanceElapsed >= curStepDistance) {
          curStepIndex++;
          if(curStepIndex < stepsDistance.length-1)
            curStepDistance += stepsDistance[curStepIndex];
        }
      }
      Log.d("Distance Elapsed", ""+distanceElapsed);
    }
  }

  @Override
  public void onAccuracyChanged(Sensor sensor, int i) {}
}

//Portions of this page are modifications based on work created and shared by Google 
//and used according to terms described in the Creative Commons 4.0 Attribution License.

\end{lstlisting}

\begin{lstlisting}[language=Java,caption=TexturedMesh.java]
package id.ac.unpar.informatika.homerunner;

import android.content.Context;
import android.opengl.GLES20;
import de.javagl.obj.Obj;
import de.javagl.obj.ObjData;
import de.javagl.obj.ObjReader;
import de.javagl.obj.ObjUtils;
import java.io.IOException;
import java.io.InputStream;
import java.nio.ByteBuffer;
import java.nio.ByteOrder;
import java.nio.FloatBuffer;
import java.nio.IntBuffer;
import java.nio.ShortBuffer;

class TexturedMesh {
  private static final String TAG = "TexturedMesh";

  private final FloatBuffer vertices;
  private final FloatBuffer uv;
  private final ShortBuffer indices;
  private final int positionAttrib;
  private final int uvAttrib;

  public TexturedMesh(Context context, String objFilePath, int positionAttrib, int uvAttrib)
      throws IOException {
    InputStream objInputStream = context.getAssets().open(objFilePath);
    Obj obj = ObjUtils.convertToRenderable(ObjReader.read(objInputStream));
    objInputStream.close();

    IntBuffer intIndices = ObjData.getFaceVertexIndices(obj, 3);
    vertices = ObjData.getVertices(obj);
    uv = ObjData.getTexCoords(obj, 2);

    indices =
        ByteBuffer.allocateDirect(2 * intIndices.limit())
            .order(ByteOrder.nativeOrder())
            .asShortBuffer();
    while (intIndices.hasRemaining()) {
      indices.put((short) intIndices.get());
    }
    indices.rewind();

    this.positionAttrib = positionAttrib;
    this.uvAttrib = uvAttrib;
  }

  public void draw() {
    GLES20.glEnableVertexAttribArray(positionAttrib);
    GLES20.glVertexAttribPointer(positionAttrib, 3, GLES20.GL_FLOAT, false, 0, vertices);
    GLES20.glEnableVertexAttribArray(uvAttrib);
    GLES20.glVertexAttribPointer(uvAttrib, 2, GLES20.GL_FLOAT, false, 0, uv);
    GLES20.glDrawElements(GLES20.GL_TRIANGLES, indices.limit(), GLES20.GL_UNSIGNED_SHORT, indices);
  }
}
}
\end{lstlisting}

\begin{lstlisting}[language=Java,caption=Util.java]

package id.ac.unpar.informatika.homerunner;

import static android.opengl.GLU.gluErrorString;

import android.opengl.GLES20;
import android.opengl.Matrix;
import android.text.TextUtils;
import android.util.Log;

class Util {
  private static final String TAG = "Util";

  private static final boolean HALT_ON_GL_ERROR = true;

  private Util() {}

  public static void checkGlError(String label) {
    int error = GLES20.glGetError();
    int lastError;
    if (error != GLES20.GL_NO_ERROR) {
      do {
        lastError = error;
        Log.e(TAG, label + ": glError " + gluErrorString(lastError));
        error = GLES20.glGetError();
      } while (error != GLES20.GL_NO_ERROR);

      if (HALT_ON_GL_ERROR) {
        throw new RuntimeException("glError " + gluErrorString(lastError));
      }
    }
  }

  public static int compileProgram(String[] vertexCode, String[] fragmentCode) {
    checkGlError("Start of compileProgram");
    
    int vertexShader = GLES20.glCreateShader(GLES20.GL_VERTEX_SHADER);
    GLES20.glShaderSource(vertexShader, TextUtils.join("\n", vertexCode));
    GLES20.glCompileShader(vertexShader);
    checkGlError("Compile vertex shader");

    int fragmentShader = GLES20.glCreateShader(GLES20.GL_FRAGMENT_SHADER);
    GLES20.glShaderSource(fragmentShader, TextUtils.join("\n", fragmentCode));
    GLES20.glCompileShader(fragmentShader);
    checkGlError("Compile fragment shader");

    int program = GLES20.glCreateProgram();
    GLES20.glAttachShader(program, vertexShader);
    GLES20.glAttachShader(program, fragmentShader);

    GLES20.glLinkProgram(program);
    int[] linkStatus = new int[1];
    GLES20.glGetProgramiv(program, GLES20.GL_LINK_STATUS, linkStatus, 0);
    if (linkStatus[0] != GLES20.GL_TRUE) {
      String errorMsg = "Unable to link shader program: \n" + GLES20.glGetProgramInfoLog(program);
      Log.e(TAG, errorMsg);
      if (HALT_ON_GL_ERROR) {
        throw new RuntimeException(errorMsg);
      }
    }
    checkGlError("End of compileProgram");

    return program;
  }

  public static float angleBetweenVectors(float[] vec1, float[] vec2) {
    float cosOfAngle = dotProduct(vec1, vec2) / (vectorNorm(vec1) * vectorNorm(vec2));
    return (float) Math.acos(Math.max(-1.0f, Math.min(1.0f, cosOfAngle)));
  }

  private static float dotProduct(float[] vec1, float[] vec2) {
    return vec1[0] * vec2[0] + vec1[1] * vec2[1] + vec1[2] * vec2[2];
  }

  private static float vectorNorm(float[] vec) {
    return Matrix.length(vec[0], vec[1], vec[2]);
  }
}
\end{lstlisting}