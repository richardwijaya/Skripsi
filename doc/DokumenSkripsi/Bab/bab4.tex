\chapter{Rancangan}
\label{chap:rancangan}
Bab ini menjelaskan perancangan aplikasi, termasuk algoritma-algoritma untuk mengolah \textit{Google StreetView API}, \textit{Google Directions API}, serta modifikasi yang dilakukan pada aplikasi HelloVR untuk membangun aplikasi \textit{jogging} virtual. 


\section{Rancangan Antarmuka}
Aplikasi terdiri atas dua halaman utama, yaitu halaman utama dan halaman VR.

\subsection{Halaman Utama}
Halaman utama adalah halaman yang dapat menerima masukan pengguna, yaitu lokasi asal dan lokasi tujuan saat berlari. Lalu, ada satu tombol untuk memicu munculnya halaman kedua. 

\subsection{Halaman VR}
Halaman VR adalah halaman yang muncul ketika pengguna sedang berlari. Tampilan \textit{Google Cardboard} VR muncul untuk menampilkan pemandangan di luar. 



\section{Rancangan Program}
Subbab ini akan menjelaskan rancangan  program, mulai dari rancangan kelas dan algoritma-algoritma yang digunakan pada \textit{method-method} yang penting. 

\subsection{Rancangan Kelas}
Rancangan kelas dari aplikasi akan menggunakan seluruh bagian pada Aplikasi HelloVR dan beberapa tambahan kelas. 


\subsection{Algoritma-Algoritma yang digunakan}
Ada beberapa algoritma yang digunakan   untuk melakukan beberapa proses seperti mengolah \textit{Google StreetView API},\textit{Google Directions API}, dan menampilkan .

\subsubsection{Mengolah \textit{Google StreetView API}} 
Sebelum mengolah gambar \textit{StreetView API}, empat gambar dari empat pandangan haruslah diunduh terlebih dahulu menggunakan HTTP/HTTPS. Setelah semua gambar itu terunduh, semua gambar itu disatukan, menjadi satu gambar untuk menjadi tekstur bangun ruang silinder. Algoritma yang digunakan untuk menyatukan gambar-gambar \textit{StreetView} tersebut adalah:

Gambar yang sudah disatukan itu akan disimpan dalam \textit{cache} perangkat. 

\subsubsection{Mengolah \textit{Google Directions API}}
Setelah \textit{file} JSON dari \textit{Directions API}, file itu diunduh ke \textit{cache} perangkat.

Kelas \textit{StreetViewLoader} adalah kelas digunakan untuk mengunduh dan mengolah gambar-gambar \textit{StreetView} untuk menghasilkan gambar yang akan dijadikan tekstur bangun ruang silinder. 

Kelas \textit{DirectionsViewLoader} adalah kelas yang digunakan untuk mengunduh file JSON yang berisi rute perjalanan sesuai masukan pengguna.

\section{\textit{Folder Assets}}
Menghapus \textit{file-file} dari aplikasi HelloVR. 

Mengganti CubeRoom.obj dengan file OBJ dengan bangun ruang silinder.

Mengganti file CubeRoom_BakedDiffuse.png, menggantinya dengan gambar \textit{StreetView} yang sudah disatukan. 


