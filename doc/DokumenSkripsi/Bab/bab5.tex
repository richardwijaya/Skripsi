\chapter{Implementasi dan Pengujian}

Pada bab ini akan dijelaskan mengenai implementasi, pengujian, dan masalah yang dihadapi. 

\section{Implementasi}
Aplikasi dikembangkan dengan Android Studio, dengan sebuah \textit{laptop} . Aplikasi dijalankan pada x buah ponsel Android.

\subsection{Lingkunan Implementasi}
Berikut adalah spesifikasi \textit{laptop} dan kakas yang digunakan untuk implementasi:

\begin{enumerate}
	\item \textit{Processor} : Intel(R) Core(TM) i7-7700HQ CPU @ 2.80GHz.
	
	\item RAM : 16.0 GB (15.9 GB usable).
	
	\item Versi Android Studio : 4.1.1
	
	\item Google VR for Android : 1.190
\end{enumerate}

Berikut adalah spesifikasi ponsel Android yang digunakan untuk implementasi:

\begin{enumerate}
	\item Processor : Qualcomm Snapdragon 835
	
	\item Sistem Operasi : Android 9.0 (Pie)
	
	\item Sensor yang dimiliki : Accelerometer, gyroscope, magnetometer, step detector.  
\end{enumerate}

\subsection{Hasil Implementasi}
Yang dihasilkan dari implementasi ini adalah aplikasi VR untuk berlari. Aplikasi dapat diperoleh di Google Play Store dan dapat di-\textit{install} pada gawai dengan sistem operasi Android. 

\subsubsection{Tampilan Awal Aplikasi ketika Dibuka}
Saat aplikasi dibuka pertama kali, aplikasi akan menampilkan dua buah \textit{textbox} dan sebuah tombol yang bertuliskan "\textit{Start Running}". 

\subsubsection{Tampilan Ketika Tombol Ditekan dengan \textit{Textbox} dalam Keadaan Kosong}
Ketika pengguna menekan tombol "\textit{Start Running}" dan keadaan salah satu \textit{textbox} kosong, akan ada \textit{Toast} yang memberikan pesan bahwa \textit{textbox} tidak boleh ada dalam keadaan kosong.

\subsubsection{Aplikasi saat Pengguna Berlari}
Setelah memasukkan masukan lokasi asal dan tujuan yang benar lewat dua \textit{textbox}, aplikasi akan menampilkan dunia VR dengan pemandangan sesuai dengan lokasi asal, dan pemandangan akan berubah sesuai langkah kaki pengguna. 



\section{Pengujian}



\section{Masalah yang Dihadapi}
Adapun masalah-masalah yang dihadapi dalam proses pengembangan aplikasi adalah sebagai berikut:

\begin{enumerate}
	\item Munculnya galat pada saat mencoba menggambar ulang bangun ruang silinder dengan tekstur baru. Galat tersebut adalah pemanggilan \textit{OpenGL renderer} pada \textit{thread} yang tidak memiliki \textit{context}. 
	
	\item Galat saat meng-\textit{update} Gradle yang membuat kemajuan pengembangan aplikasi terhambat. 
\end{enumerate}

