  %versi 2 (8-10-2016) 
\chapter{Pendahuluan}
\label{chap:intro}
   
\section{Latar Belakang}
\label{sec:label}

Zaman modern adalah zaman saat profesi sudah dan sedang berkembang sehingga ada banyak sekali jumlah bidangnya serta perkembangannya. Mayoritas orang menekuni bidang-bidang profesi yang tak terhitung banyaknya untuk mengembangkan setiap bidang profesi. Hal ini menyebabkan kesulitan pengaturan waktu untuk berolahraga, yang adalah salah satu kebutuhan manusia untuk menjaga kesehatan. Salah satu aktivitas olahraga yang paling mudah dan tidak memerlukan gerakan yang sulit adalah berlari, namun kegiatan ini memerlukan lahan yang cukup besar agar dapat dilakukan dengan leluasa. Selain kebutuhan lahan, aktivitas berlari lebih menyenangkan jika dilakukan di luar rumah. Agar dapat dilakukan di dalam rumah, berlari dapat dilakukan di rumah adalah {\it treadmill}, akan tetapi masalah lingkungan yang monoton dan membosankan di dalam rumah membuat orang enggan untuk melakukan aktivitas berlari. Bila suasana dunia luar dapat dibawa ke dalam rumah, aktivitas ini dapat dilakukan di dalam rumah, tetapi suasana yang dirasakan adalah seperti di luar rumah. 

Pada skripsi ini, akan dibuat sebuah perangkat lunak yang dapat menampilkan simulasi aktivitas berlari pada lingkungan yang diinginkan saat berlari di {\it treadmill}. Dengan menggunakan perangkat lunak tersebut, orang yang berlari dapat menikmati pemandangan yang dipilih saat berlari di dalam rumah sehingga merasa seperti berlari di lingkungan yang dipilih tersebut.

Teknologi yang dapat dimanfaatkan untuk membuat aplikasi VR untuk berlari adalah \textit{Google Cardboard} dan sensor perangkat bergerak, dan untuk {\it Application Programming Interface} (\textit{API}) yang digunakan adalah {\it Google Streetview API}  dan \textit{Google Directions API} ~\cite{quickstart-google-vr}~\cite{streetview-api}~\cite{directions-api}. 

\section{Rumusan Masalah}
\label{sec:rumusan}
Rumusan masalah yang ada pada skripsi ini adalah:
\begin{itemize}
	\item Bagaimana memanfaatkan Google VR SDK for Android untuk menampilkan gambar dengan perangkat VR?
	\item Bagaimana menampilkan hasil dari \textit{Google StreetView API} dalam bentuk VR?
	\item Bagaimana mengintegrasikan \textit{Google Directions API}, gambar \textit{Google StreetView}  dan Google VR dalam perangkat lunak \textit{virtual jogging}
\end{itemize}

%\dtext{6}

\section{Tujuan}
\label{sec:tujuan}
Pada skripsi ini, hal-hal yang coba untuk dicapai adalah :
\begin{itemize}
	\item Menggunakan Google VR SDK for Android untuk menampilkan gambar dengan {\it Google Cardboard}.
	\item Menampilkan hasil gambar dari \textit{Google StreetView API} pada {\it Google Cardboard}.
	\item Mengintegrasikan \textit{Google Directions API}, gambar dari \textit{Google StreetView} dan Google VR (\textit{Cardboard}) dalam perangkat lunak {\it virtual jogging}.
\end{itemize}
%Akan dipaparkan secara lebih terperinci dan tersturkur apa yang menjadi tujuan pembuatan template skripsi ini

%\dtext{7}

\section{Batasan Masalah}
\label{sec:batasan}
Karena ada banyak tempat atau pemandangan di dunia ini, hanya beberapa lokasi saja yang dapat dipilih dari yang telah disediakan.
%Untuk mempermudah pembuatan template ini, tentu ada hal-hal yang harus dibatasi, misalnya saja bahwa template ini bukan berupa style \LaTeX{} pada umumnya (dengan alasannya karena belum mampu jika diminta membuat seperti itu)

%\dtext{8}

\section{Metodologi Penelitian}
\label{sec:metlit}
Metodologi penelitian yang akan digunakan adalah sebagai berikut:
\begin{itemize}
	\item Melakukan studi literatur dari situs-situs web tentang Google VR SDK, \textit{StreetView API}, \textit{Directions}, sensor tentang langkah, baik melalui media tulisan maupun video.
	\item Menampilkan pemandangan {\it StreetView} pada {\it Google Cardboard}s.
	\item Mengintegrasikan {\it Google Directions API} dengan pemandangan {\it StreetView} yang telah ditampilkan pada {\it Google Cardboard}.
	\item Menganalisis sensor langkah dan menyinkornisasikannya dengan perubahan pemandangan \textit{StreetView}.
\end{itemize}
Untuk membuat skripsi ini, peneliti . Setelah mempelajari semua komponen dari aplikasi yang akan dibuat, peneliti akan melakukan implementasi. 

\section{Sistematika Pembahasan}
\label{sec:sispem}
Dokumen dibagi ke dalam beberapa bab dengan sistematika pembahasan sebagai berikut:
\begin{enumerate}
	\item Bab 1: Pendahuluan, yang menjelaskan gambaran umum penelitian. Mengandung latar belakang, rumusan masalah, tujuan,, batasan masalah, metodologi penelitian, serta sistematika pembahasan.
	\item Bab 2: Dasar Teori, berisi landasan dari teori-teori yang berhubungan serta mendukung penelitian. Mengandung Google VR, Google {\it StreetView API}, Google {\it Directions API}, dan sensor.
	\item Bab 3: Analisis, menjelaskan mengenai proses analisis masalah untuk menemukan solusi untuk menyelesaikan masalah. Mengandung cara membuat dunia VR, cara memanfaatkan \textit{Google StreetView}, cara memanfaatkan \textit{Google Directions API}, dan cara memanfaatkan sensor \textit{step-detector}. 
	%Berisi tentang ...
\end{enumerate}
%Rencananya Bab 2 akan berisi petunjuk penggunaan template dan dasar-dasar \LaTeX.
%Mungkin bab 3,4,5 dapt diisi oleh ketiga jurusan, misalnya peraturan dasar skripsi atau pedoman penulisan, tentu jika berkenan.
%Bab 6 akan diisi dengan kesimpulan, bahwa membuat template ini ternyata sungguh menghabiskan banyak waktu.
